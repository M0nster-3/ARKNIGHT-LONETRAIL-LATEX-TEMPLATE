\section{\#01 预备知识}
\subsection{二级标题}
测试测试测试测试测试测试测试测试测试测试测试测试测试测试测试测试测试测试测试测试测试测试测试测试测试测试测试测试测试测试测试测试测试测试测试测试测试测试测试测试测试测试测试测试测试测试测试测试测试测试测试测试测试测试测试测试测试测试测试测试测试测试

\begin{definition}{测试定义}
	测试一下测试一下
\end{definition}

\begin{theorem}{测试定理}
	测试一下测试一下
\end{theorem}

\begin{proposition}{测试命题}
	测试一下测试一下
\end{proposition}

\begin{corollary}{测试推论}
	测试一下测试一下
\end{corollary}

\begin{lemma}{测试引理}
	测试一下测试一下
\end{lemma}

\begin{example}{测试示例}
	测试一下测试一下
\end{example}

\begin{exercise}{练习}
	证明练习题
\end{exercise}

\begin{prove}
	好好好测试测试测试好好好测试测试测试好好好测试测试测试好好好测试测试测试好好好测试测试测试好好好测试测试测试好好好测试
\end{prove}

\begin{remark}
	测试一下
\end{remark}


\newpage
\begin{box1}{测试}
	测试测试测试
\end{box1}

\begin{box2}{测试}
	测试测试测试
\end{box2}

\begin{box3}{测试}
	测试测试测试
\end{box3}

\begin{box4}{测试}
	测试测试测试
\end{box4}
\newpage
\subsection{序数}
\newpage
\subsection{基数}
\newpage
\subsection{本章练习}
\newpage

\section{\#02 树}
\subsection{基本概念}
\newpage
\subsection{树与闭集}
\newpage
\subsection{乘积集上的树}
\newpage
\subsection{最左分支}
\newpage
\subsection{良基树与秩}
\newpage
\subsection{树的良基部分}
\newpage
\subsection{Kleene-Brouwer序}
\newpage
\subsection{本章练习}
\newpage

\section{\#03 波兰空间}
\subsection{相关定义和例子}
\newpage
\subsection{连续延拓和同胚延拓}
\newpage
\subsection{波兰空间的子波兰空间}
\newpage
\subsection{本章练习}
\newpage

\section{\#04 紧度量空间}
\subsection{基本事实}
\newpage
\subsection{相关例子}
\newpage
\subsection{Hilbert立方体的万有性质}
\newpage
\subsection{Cantor空间的连续像}
\newpage
\subsection{紧空间上的连续函数空间}
\newpage
\subsection{紧集的超空间}
\newpage
\subsection{本章练习}
\newpage

\section{\#05 局部紧空间}
\subsection{基本事实}
\newpage

\section{\#06 完美波兰空间}
\subsection{Cantor空间的嵌入}
\newpage
\subsection{Cantor-Bendixson定理}
\newpage
\subsection{Cantor-Bendixso导子与秩}
\newpage
\subsection{本章练习}
\newpage

\section{\#07 零维空间}
\subsection{基本事实}
\newpage
\subsection{Cantor空间的拓扑刻画}
\newpage
\subsection{Baire空间的拓扑刻画}
\newpage
\subsection{零维空间到Baire空间的嵌入}
\newpage
\subsection{波兰空间作为Baire空间的连续像}
\newpage
\subsection{与Baire空间同胚的闭子集}
\newpage
\subsection{本章练习}
\newpage

\section{\#08 Baire纲}
\subsection{贫集}
\newpage