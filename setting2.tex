%通用盒子定义===============================================================================
\newlength{\textwd} %自动计算文本长度

% 通用三栏标题盒命令1(左右图片 + 中间自适应文字)
\newcommand{\TitleBoxI}[5]{%
	\setlength{\textwd}{0pt}%
	\sbox0{\fontsize{20}{22}\selectfont\SiYuan\XeTeXinterchartokenstate=0 #2}%
	\setlength{\textwd}{\widthof{\usebox0}}%
	\begin{tabular}[c]{@{}c@{}c@{}c@{}}
		\raisebox{-\height + 12pt}{%
			\resizebox{!}{24pt}{\includegraphics{#1}}%
		} &
		\raisebox{-\height + 12pt}{%
			\colorbox{#4}{%
				\parbox[c][18pt][c]{\textwd}{%
					\centering
					\fontsize{18}{22}\selectfont\SiYuan\XeTeXinterchartokenstate=0\color{#5} #2%
				}%
			}%
		} &
		\raisebox{-\height + 12pt}{%
			\resizebox{!}{24pt}{\includegraphics{#3}}%
		}
	\end{tabular}%
}

% 通用三栏标题盒命令2(左右图片 + 中间自适应文字)
\newcommand{\TitleBoxII}[5]{%
	\setlength{\textwd}{0pt}%
	\sbox0{\fontsize{16}{22}\selectfont\SiYuan\XeTeXinterchartokenstate=0 #2}%
	\setlength{\textwd}{\widthof{\usebox0}}%
	\begin{tabular}[c]{@{}c@{}c@{}c@{}}
		\raisebox{-\height + 12pt}{%
			\resizebox{!}{24pt}{\includegraphics{#1}}%
		} &
		\raisebox{-\height + 12pt}{%
			\colorbox{#4}{%
				\parbox[c][18pt][c]{\textwd}{%
					\centering
					\fontsize{14}{22}\selectfont\SiYuan\XeTeXinterchartokenstate=0\color{#5} #2%
					\vspace{-4pt}
				}%
			}%
		} &
		\raisebox{-\height + 12pt}{%
			\resizebox{!}{24pt}{\includegraphics{#3}}%
		}
	\end{tabular}%
}

% 通用三栏标题盒命令3(左右图片 + 中间自适应文字)
\newcommand{\TitleBoxIII}[5]{%
	\setlength{\textwd}{0pt}%
	\sbox0{\fontsize{22}{22}\selectfont\SiYuan\XeTeXinterchartokenstate=0 #2}%
	\setlength{\textwd}{\widthof{\usebox0}}%
	\begin{tabular}[c]{@{}c@{}c@{}c@{}}
		\raisebox{-\height + 12pt}{%
			\resizebox{!}{24pt}{\includegraphics{#1}}%
		} &
		\raisebox{-\height + 12pt}{%
			\colorbox{#4}{%
				\parbox[c][18pt][c]{\textwd}{%
					\centering
					\fontsize{14}{22}\selectfont\SiYuan\XeTeXinterchartokenstate=0\color{#5} #2%
					\vspace{0pt}
				}%
			}%
		} &
		\raisebox{-\height + 12pt}{%
			\resizebox{!}{24pt}{\includegraphics{#3}}%
		}
	\end{tabular}%
}

% 通用三栏标题盒命令4(左右图片 + 中间自适应文字)
\newcommand{\TitleBoxIV}[5]{%
	\setlength{\textwd}{0pt}%
	\sbox0{\fontsize{16}{22}\selectfont\SiYuan\XeTeXinterchartokenstate=0 #2}%
	\setlength{\textwd}{\widthof{\usebox0}}%
	\begin{tabular}[c]{@{}c@{}c@{}c@{}}
		\raisebox{-\height + 12pt}{%
			\resizebox{!}{24pt}{\includegraphics{#1}}%
		} &
		\raisebox{-\height + 12pt}{%
			\colorbox{#4}{%
				\parbox[c][18pt][c]{\textwd}{%
					\centering
					\fontsize{14}{22}\selectfont\SiYuan\XeTeXinterchartokenstate=0\color{#5} #2%
					\vspace{0pt}
				}%
			}%
		} &
		\raisebox{-\height + 12pt}{%
			\resizebox{!}{24pt}{\includegraphics{#3}}%
		}
	\end{tabular}%
}

%相关盒子定义============================================================
% 1. 一级标题和二级标题盒子设置
\titleformat{\subsection}
{\raggedright}{}{0pt}{
	\TitleBoxI{素材库/正文/box1_1}{#1}{素材库/正文/box1_2}{mibai}{qiancheng}
	\vspace{8pt}
}

% 2. 相关数学内容盒子设置
\newtcolorbox{definition}[1]{
	enhanced,
	sharp corners,
	colback=qianbai,
	colframe=definition,
	boxrule=2.5pt,
	fontupper={\kaishu},
	halign=justify,
	breakable,
	title={\TitleBoxII{素材库/正文/definition_1}{#1}{素材库/正文/definition_2}{heise}{mibai}},
	attach boxed title to top left={yshift=-2.5pt, xshift=0mm},
	boxed title style={
		boxrule=0pt,
		boxsep=0pt,
		top=-2.1pt,
		bottom=0pt,
		left=0pt,
		right=0pt,
		colback=qianbai,
		baseline=\ht\strutbox
	},
	after upper={\hfill\includegraphics[height=8pt]{素材库/正文/definition_3}},
}

\newtcolorbox{theorem}[1]{
	enhanced,
	sharp corners,
	colback=qianbai,
	colframe=theorem,
	boxrule=2.5pt,
	fontupper={\kaishu},
	halign=justify,
	breakable,
	title={\TitleBoxII{素材库/正文/theorem_1}{#1}{素材库/正文/theorem_2}{heise}{mibai}},
	attach boxed title to top left={yshift=-2.5pt, xshift=0mm},
	boxed title style={
		boxrule=0pt,
		boxsep=0pt,
		top=-2.1pt,
		bottom=0pt,
		left=0pt,
		right=0pt,
		colback=qianbai,
		baseline=\ht\strutbox
	},
	after upper={\hfill\includegraphics[height=8pt]{素材库/正文/theorem_3}},
}

\newtcolorbox{proposition}[1]{
	enhanced,
	sharp corners,
	colback=qianbai,
	colframe=proposition,
	boxrule=2.5pt,
	fontupper={\kaishu},
	halign=justify,
	breakable,
	title={\TitleBoxII{素材库/正文/proposition_1}{#1}{素材库/正文/proposition_2}{heise}{mibai}},
	attach boxed title to top left={yshift=-2.5pt, xshift=0mm},
	boxed title style={
		boxrule=0pt,
		boxsep=0pt,
		top=-2.1pt,
		bottom=0pt,
		left=0pt,
		right=0pt,
		colback=qianbai,
		baseline=\ht\strutbox
	},
	after upper={\hfill\includegraphics[height=8pt]{素材库/正文/proposition_3}},
}

\newtcolorbox{corollary}[1]{
	enhanced,
	sharp corners,
	colback=qianbai,
	colframe=corollary,
	boxrule=2.5pt,
	fontupper={\kaishu},
	halign=justify,
	breakable,
	title={\TitleBoxII{素材库/正文/corollary_1}{#1}{素材库/正文/corollary_2}{heise}{mibai}},
	attach boxed title to top left={yshift=-2.5pt, xshift=0mm},
	boxed title style={
		boxrule=0pt,
		boxsep=0pt,
		top=-2.1pt,
		bottom=0pt,
		left=0pt,
		right=0pt,
		colback=qianbai,
		baseline=\ht\strutbox
	},
	after upper={\hfill\includegraphics[height=8pt]{素材库/正文/corollary_3}},
}

\newtcolorbox{lemma}[1]{
	enhanced,
	sharp corners,
	colback=qianbai,
	colframe=lemma,
	boxrule=2.5pt,
	fontupper={\kaishu},
	halign=justify,
	breakable,
	title={\TitleBoxII{素材库/正文/lemma_1}{#1}{素材库/正文/lemma_2}{heise}{mibai}},
	attach boxed title to top left={yshift=-2.5pt, xshift=0mm},
	boxed title style={
		boxrule=0pt,
		boxsep=0pt,
		top=-2.1pt,
		bottom=0pt,
		left=0pt,
		right=0pt,
		colback=qianbai,
		baseline=\ht\strutbox
	},
	after upper={\hfill\includegraphics[height=8pt]{素材库/正文/lemma_3}},
}

\newtcolorbox{example}[1]{
	enhanced,
	sharp corners,
	colback=qianbai,
	colframe=example,
	boxrule=2.5pt,
	fontupper={\kaishu},
	halign=justify,
	breakable,
	title={\TitleBoxII{素材库/正文/example_1}{#1}{素材库/正文/example_2}{heise}{mibai}},
	attach boxed title to top left={yshift=-2.5pt, xshift=0mm},
	boxed title style={
		boxrule=0pt,
		boxsep=0pt,
		top=-2.1pt,
		bottom=0pt,
		left=0pt,
		right=0pt,
		colback=qianbai,
		baseline=\ht\strutbox
	},
	after upper={\hfill\includegraphics[height=8pt]{素材库/正文/example_3}},
}

\newtcolorbox{exercise}[1]{
	enhanced,
	sharp corners,
	colback=qianbai,
	colframe=qiancheng,
	boxrule=2.5pt,
	fontupper={\kaishu},
	halign=justify,
	breakable,
	title={\TitleBoxII{素材库/正文/exercise_1}{#1}{素材库/正文/exercise_2}{mibai}{qiancheng}},
	attach boxed title to top left={yshift=0pt, xshift=0mm},
	boxed title style={
		boxrule=0pt,
		boxsep=0pt,
		top=-2.1pt,
		bottom=0pt,
		left=0pt,
		right=0pt,
		colback=qianbai,
		baseline=\ht\strutbox
	},
}

\newtcolorbox{question}[1]{
	enhanced,
	sharp corners,
	colback=qianbai,
	colframe=qiancheng,
	boxrule=2.5pt,
	fontupper={\kaishu},
	halign=justify,
	breakable,
	title={\TitleBoxII{素材库/正文/question_1}{#1}{素材库/正文/question_2}{mibai}{qiancheng}},
	attach boxed title to top left={yshift=0pt, xshift=0mm},
	boxed title style={
		boxrule=0pt,
		boxsep=0pt,
		top=-2.1pt,
		bottom=0pt,
		left=0pt,
		right=0pt,
		colback=qianbai,
		baseline=\ht\strutbox
	},
}

\newtcolorbox{prove}[1][]{
	enhanced,
	sharp corners,
	colback=qianbai,  
	colframe=qiancheng,  
	fontupper={\kaishu}, 
	halign=justify,
	leftrule=5pt, 
	rightrule=0pt,      
	toprule=0pt,        
	bottomrule=0pt,     
	breakable,
	before upper={
		\setstretch{1.5}\noindent
		\textcolor{qiancheng}{\XeTeXinterchartokenstate=0\SiYuan\textbf{证明:}} 
	},
	after upper={\hfill\textcolor{qiancheng}{$\blacksquare$}},
	#1 
}

\newtcolorbox{remark}[1][]{
	enhanced,
	sharp corners,
	colback=qianbai,  
	colframe=qiancheng,  
	fontupper={\kaishu},  
	halign=justify,
	leftrule=5pt, 
	rightrule=0pt,      
	toprule=0pt,        
	bottomrule=0pt,     
	breakable,
	before upper={
		\setstretch{1.5}\noindent
		\textcolor{qiancheng}{\XeTeXinterchartokenstate=0\SiYuan\textbf{点评:}} 
	},
	after upper={\hfill\textcolor{qiancheng}{$\blacksquare$}},
	#1 
}

% 3. 其他盒子设置
\newtcolorbox{box1}[1]{
	enhanced,
	sharp corners,
	colback=qianbai,
	colframe=qiancheng,
	fontupper={\kaishu},
	halign=justify,
	boxrule=2pt,
	breakable,
	title={\TitleBoxIII{素材库/正文/box1_1}{#1}{素材库/正文/box1_2}{mibai}{qiancheng}},
	attach boxed title to top left={yshift=0pt, xshift=0mm},
	boxed title style={
		boxrule=0pt,
		boxsep=0pt,
		top=-2.1pt,
		bottom=0pt,
		left=0pt,
		right=0pt,
		colback=qianbai,
		baseline=\ht\strutbox
	},
}

\newtcolorbox{box2}[1]{
	enhanced,
	sharp corners,
	colback=qianbai,
	colframe=qiancheng,
	fontupper={\kaishu},
	halign=justify,
	boxrule=2pt,
	breakable,
	title={\TitleBoxIV{素材库/正文/box2_1}{#1}{素材库/正文/box2_2}{qiancheng}{mibai}},
	attach boxed title to top left={yshift=0pt, xshift=0mm},
	boxed title style={
		boxrule=0pt,
		boxsep=0pt,
		top=-2.1pt,
		bottom=0pt,
		left=0pt,
		right=0pt,
		colback=qianbai,
		baseline=\ht\strutbox
	},
}

\newtcolorbox{box3}[1]{
	enhanced,
	sharp corners,
	colback=qianbai,
	colframe=qiancheng,
	fontupper={\kaishu},
	halign=justify,
	boxrule=2pt,
	breakable,
	title={\TitleBoxIII{素材库/正文/box3_1}{#1}{素材库/正文/box3_2}{heise}{mibai}},
	attach boxed title to top left={yshift=0pt, xshift=0mm},
	boxed title style={
		boxrule=0pt,
		boxsep=0pt,
		top=-2.1pt,
		bottom=0pt,
		left=0pt,
		right=0pt,
		colback=qianbai,
		baseline=\ht\strutbox
	},
}

\newtcolorbox{box4}[1]{
	enhanced,
	sharp corners,
	colback=qianbai,
	colframe=qiancheng,
	fontupper={\kaishu},
	halign=justify,
	boxrule=2pt,
	breakable,
	title={\TitleBoxIV{素材库/正文/box4_1}{#1}{素材库/正文/box4_2}{mibai}{qiancheng}},
	attach boxed title to top left={yshift=0pt, xshift=0mm},
	boxed title style={
		boxrule=0pt,
		boxsep=0pt,
		top=-2.1pt,
		bottom=0pt,
		left=0pt,
		right=0pt,
		colback=qianbai,
		baseline=\ht\strutbox
	},
}


%自定义其他命令===================================================
% 1. 自定义分割线
\newcommand{\Ark}[0]{
	\par\vspace{-10pt}
	\noindent
	\includegraphics[width=\linewidth, height=2pt]{素材库/正文/separator_1}
	\par\vspace{-4pt}
}

% 2. 自定义文字居中
\newcommand{\C}[1]{\makebox[\linewidth][c]{#1}}

% 3. 自定义增加间隔
\newcommand{\Q}[0]{\vspace{4pt}}

% 4. 自定义添加文本背景
\newcommand{\B}[1]{%
	\tikz[baseline=(char.base)]{%
		\node[inner sep=0pt, outer sep=0pt] (char) {\phantom{#1}};
		\fill[qiancheng] (char.south west) rectangle ([xshift=0pt, yshift=1.5pt]char.south east);
		\node[inner sep=0pt, outer sep=0pt] (char) {#1};
	}%
}